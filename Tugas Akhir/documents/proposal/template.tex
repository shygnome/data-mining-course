\documentclass{article}
\usepackage[utf8]{inputenc}
\usepackage[margin=0.75in]{geometry}
\usepackage{enumitem}
\usepackage{siunitx}

\setlength{\parskip}{1em}

\begin{document}
    \begin{center}
        
		\LARGE{\textbf{Proposal Tugas Akhir}} \\
        \vspace{1em}
        \Large{\textit{PM for PM2.5}: Analisis pada Dataset \textit{Air Quality}} \\
        \vspace{1em}
        \normalsize\textbf{Andrew Savero Ongko, Muhammad Fakhrillah Abdul Azis,} \\
        \normalsize\textbf{Muhammad Yudistira Hanifmuti} \\
        \normalsize{\{andrew.savero, muhammad.fakhrillah, muhammad.yudistira\}@ui.ac.id} \\
        \vspace{1em}
        \normalsize{Universitas Indonesia, Depok} \\
        \normalsize{Fakultas Ilmu Komputer}

	\end{center}
    \begin{normalsize}
        
        \section{Identifikasi Masalah}
        Proposal ini dimulai dengan identifikasi dari masalah yang ingin diselesaikan. 
        Pada kesempatan ini, kelompok kami mendapatkan dataset \textit{Air Quality} 
        untuk pengerjaan Tugas Akhir Mata Kuliah Penambangan Data. Dataset tersebut memiliki 
        sebuah atribut target bernama "PM 2.5" dan beberapa atribut lain yang merupakan hasil 
        pengawasan kualitas udara oleh beberapa stasiun pengawasan di suatu negara. Atribut 
        PM2.5 merujuk pada istilah \textit{"Atmospheric Particulate Matter"} yang memiliki
        diameter $<$ \SI{2.5}{\micro\metre}. Ukurannya yang kecil membuat partikel ini dapat 
        masuk jauh ke dalam sistem pernafasan dan peredaran darah yang dapat menyebabkan 
        serangan jantung, penyakit pernafasan, hingga kematian. Sehingga konsentrasi PM2.5
        ini cukup penting untuk dimonitor dan diproses lebih lanjut. Hal ini dapat membantu 
        mengetahui tempat dan waktu krusial yang memiliki tingkat PM2.5 tinggi agar bisa
        dilakukan usaha pencegahan untuk menghindari hal-hal yang tidak diinginkan.

        \noindent Seperti yang disinggung pada paragraf sebelumnya, dataset \textit{Air Quality}
        berisikan atribut yang menjelaskan kondisi kualitas udara, seperti suhu dan tekanan udara,
        pada suatu waktu di suatu tempat tertentu. Pada setiap baris pada dataset, terdapat kolom
        PM2.5 yang menyatakan konsentrasi partikel PM2.5 yang ada di udara dalam satuan 
        (ug/\SI{}{\metre\cubed}). Kami akan mencoba membuat model \textit{machine learning} yang 
        dapat mempelajari pola kualitas udara agar bisa digunakan untuk memprediksi kadar PM2.5 
        di udara. Metode \textit{data mining} yang digunakan adalah \textit{Regression} karena 
        nilai kadar PM2.5 yang menjadi target bersifat kontinu (ug/\SI{}{\metre\cubed}). Oleh
        karena itu, kami akan menerapkan metode dan langkah-langkah yang biasa digunakan untuk
        membangun model \textit{Regression}.

        \section{Hasil Explorasi Data}
        
        Hasil Explorasi

        \section{Desain Cara Kerja}

        Desain Cara Kerja

        1 Data Cleansing Process
        Data cleansing is the first and a very crucial step in the overall data preparation
        process and is the process of analysing, identifying and correcting messy, raw data.
        When analysing organisational data to make strategic decisions you must start with
        a thorough data cleansing process.

        \begin{enumerate}[label=\alph*.]
            \item We need to remove unused columns such as
            \item We need to remove unused columns such as  
        \end{enumerate}
        a. We need to remove unused columns such as
        b. Removing the duplicacy in the rows
        c. Delete values have not been recorded or some information is missing
        d. Changing release date column into Date format
        e. 

        2 Identifying relationships between variables / features
        The main goal here is to identify and create relationships which can help you to build a hypothesis. 
        We’ll have to define questions which can help us build some relationships to look at.
        
        \section{Metode yang Digunakan}

        Metode yang digunakan

        \section{Referensi}

        Referensi
        https://towardsdatascience.com/hitchhikers-guide-to-exploratory-data-analysis-6e8d896d3f7e
    \end{normalsize}
  
\end{document}