\documentclass{article}
\usepackage[utf8]{inputenc}
\usepackage[margin=0.75in]{geometry}

\begin{document}
	\begin{center}
    
    	% MAKE SURE YOU TAKE OUT THE SQUARE BRACKETS
    
		\LARGE{\textbf{Proposal Tugas Akhir}} \\
        \vspace{1em}
        \Large{Complete or Obsolete: A re-evaluation of research} \\
        \vspace{1em}
        \normalsize\textbf{Joseph Eppich} \\
        \normalsize{Eppichj@sunypoly.edu} \\
        \vspace{1em}
        \normalsize{Advisor: Joshua White} \\
        \vspace{1em}
        \normalsize{State University of New York Polytechnic Institute, Utica NY} \\
        \normalsize{Bachelors of Science in Network and Computer Security}
     
	\end{center}
    \begin{normalsize}
    
        \section{Catatan:}
        
        Delivery Tahap 1 - Proposal dan Exploratory Data Analysis
        Proposal dikumpulkan dalam format PDF pada tanggal 23 November 2019 pukul 20.00
        dengan memuat informasi sebagai berikut.
        a. Penjelasan singkat terkait masalah yang ingin Anda pecahkan dari dataset yang
        digunakan, termasuk mengenai analisis metode Data Mining (Clustering,
        Regression, atau classification) yang paling tepat untuk data tersebut.
        b. Penjelasan terkait hasil Exploratory Data Analysis yang Anda lakukan.
        c. Desain cara kerja dan tahapan-tahapan dalam mengelola data.
        d. Metode yang akan digunakan, termasuk alasan pemilihan metode tersebut dan
        state of the art dari metode tersebut
        
        \section{Identifikasi Masalah}
        
        Identifikasi masalah.
        
        \section{Metode yang Digunakan}

        Metode yang digunakan

        \section{Desain Cara Kerja}

        Desain Cara Kerja

        1# Data Cleansing Process
        #Data cleansing is the first and a very crucial step in the overall data preparation
        #process and is the process of analysing, identifying and correcting messy, raw data.
        #When analysing organisational data to make strategic decisions you must start with
        #a thorough data cleansing process.

        a. We need to remove unused columns such as
        b. Removing the duplicacy in the rows
        c. Delete values have not been recorded or some information is missing
        d. Changing release date column into Date format
        e. 

        #2 Identifying relationships between variables / features
        #The main goal here is to identify and create relationships which can help you to build a hypothesis. 
        #We’ll have to define questions which can help us build some relationships to look at.

	   	\section{Hasil Explorasi Data}
        
        Hasil Explorasi
        
    	\section{Timeline:}
        
        \noindent You can create bullets by doing:
        
        \begin{itemize}
            \item(2/25)have a basic idea and outline for my paper
            \item (3/1) have my capstone proposal made and submitted, and begin researching
            \item(4/19) have my paper finished submitted for final review
            \item(5/1) proof read and look over all my material and prepare for my presentation
        \end{itemize}
        
    	\section{Possible Issues:}
        Possible problems I foresee with this capstone is my lack of hardware compared to that used in the original content. beyond that technical issues with suricata and snort could lead to issues in the long run.

        \section{Referensi}

        Referensi
        https://towardsdatascience.com/hitchhikers-guide-to-exploratory-data-analysis-6e8d896d3f7e
    \end{normalsize}
  
\end{document}