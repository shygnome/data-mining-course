\documentclass{article}
\usepackage[utf8]{inputenc}
\usepackage[margin=0.75in]{geometry}

\begin{document}
	\begin{center}
    
    	% MAKE SURE YOU TAKE OUT THE SQUARE BRACKETS
    
		\LARGE{\textbf{Proposal Tugas Akhir}} \\
        \vspace{1em}
        \Large{\textit{PM for PM2.5}: Analisis pada Dataset \textit{Air Quality}} \\
        \vspace{1em}
        \normalsize\textbf{Andrew Savero Ongko, Muhammad Fakhrillah Abdul Azis,} \\
        \normalsize\textbf{Muhammad Yudistira Hanifmuti} \\
        \normalsize{\{andrew.savero, muhammad.fakhrillah, } \\
        \normalsize{muhammad.yudistira\}@ui.ac.id} \\
        \vspace{1em}
        \normalsize{Universitas Indonesia, Depok} \\
        \normalsize{Fakultas Ilmu Komputer}
     
	\end{center}
    \begin{normalsize}
    
        \section{Catatan:}
        
        Delivery Tahap 1 - Proposal dan Exploratory Data Analysis
        Proposal dikumpulkan dalam format PDF pada tanggal 23 November 2019 pukul 20.00
        dengan memuat informasi sebagai berikut.
        a. Penjelasan singkat terkait masalah yang ingin Anda pecahkan dari dataset yang
        digunakan, termasuk mengenai analisis metode Data Mining (Clustering,
        Regression, atau classification) yang paling tepat untuk data tersebut.
        b. Penjelasan terkait hasil Exploratory Data Analysis yang Anda lakukan.
        c. Desain cara kerja dan tahapan-tahapan dalam mengelola data.
        d. Metode yang akan digunakan, termasuk alasan pemilihan metode tersebut dan
        state of the art dari metode tersebut
        
        \section{Identifikasi Masalah}
        
        PM2.5 merujuk pada "atmospheric particulate matter" yang memiliki diameter < 2.5 micrometer.
        Karena ukurannya yang kecil partikel ini dapat masuk jauh ke dalam sistem pernafasan, dan
        peredaran darah yang dapat menyebabkan serangan jantung, penyakit pernafasan, hingga kematian.
        Partikel partikel tersebut muncul secara alami dari hal hal seperti gunung berapi dan kebakaran
        hutan. Namun aktivitas manusia seperti pembakaran bahan bakar fosil dan proses industri lainnya
        juga dapat meproduksi partikel partikel tersebut.

        Diberikan dataset berisikan status keadaan udara seperti suhu dan tekanan udara
        pada suatu waktu di suatu tempat. Pada setiap row pada dataset tersebut terdapat kolom
        "PM2.5" yang menyatakan konsentrasi partikel PM2.5 yang ada di udara dalam satuan (ug/m3).
        Dari dataset ini kami akan mencoba membuat model machine learning yang dapat mempelajari pola
        kualitas udara agar bisa digunakan untuk memprediksi kadar PM2.5 di udara. Hal ini dapat
        membantu mengetahui tempat dan waktu krusial yang memiliki tingkat PM2.5 tinggi agar bisa
        dilakukan usaha pencegahan agar udara pada tempat dan waktu tersebut tidak dihirup oleh masyarakat.

        Karena nilai kadar PM2.5 bersifat kontinu (ug/m3) maka kami menyimpulkan bahwa permasalahan ini
        dapat dimodelkan menggunakan regresi. Kami akan melakukan training terhadap dataset untuk menentukan
        bobot pada setiap kolom beserta nilai bias nya. Bobot yang telah didapatkan akan digunakan untuk
        memprediksi kada PM2.5 berdasarkan kualitas udaranya.
        
        \section{Metode yang Digunakan}

        Metode yang digunakan

        \section{Desain Cara Kerja}

        Desain Cara Kerja
        
        \section{Hasil Explorasi Data}
        
        Hasil Explorasi
        
    	\section{Timeline:}
        
        \noindent You can create bullets by doing:
        
        \begin{itemize}
            \item(2/25)have a basic idea and outline for my paper
            \item (3/1) have my capstone proposal made and submitted, and begin researching
            \item(4/19) have my paper finished submitted for final review
            \item(5/1) proof read and look over all my material and prepare for my presentation
        \end{itemize}
        
    	\section{Possible Issues:}
        Possible problems I foresee with this capstone is my lack of hardware compared to that used in the original content. beyond that technical issues with suricata and snort could lead to issues in the long run.

        \section{Referensi}

        Referensi
    \end{normalsize}
  
\end{document}